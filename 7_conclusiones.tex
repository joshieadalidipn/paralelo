\section{Conclusiones}
\subsection{Juárez Botello Josué Adalid}

La verdad es que intentar resolver la serie de Fourier para una función como \(f(x)=x^4-5x^2-2x+1\) fue algo que al principio me parecía súper complicado. Al principio, ni siquiera estaba seguro de por dónde empezar, pero después de meterme más en el tema y ver cómo se aplicaba paso a paso, empecé a entender un poco más cómo funcionaba.

Lo que me llamó la atención fue darme cuenta de que el análisis de Fourier no es solo un montón de ecuaciones matemáticas sin sentido. Al contrario, es algo que usamos en la vida real, y bastante. Por ejemplo, cuando hablamos sobre cómo se procesan las señales en nuestros teléfonos o cómo se pueden mejorar las imágenes que tomamos con la cámara, todo esto tiene que ver con el análisis de Fourier. Honestamente, no tenía ni idea de que algo que aprendíamos en clase se usaba para tantas cosas importantes.

Calcular los coeficientes \(a_0\), \(a_n\), y \(b_n\) fue bastante difícil. Aunque al final tuve que apoyarme en algunas herramientas digitales para sacar los cálculos más difíciles, hacerlo me ayudó a ver cómo se descomponen las funciones en partes más simples que puedes analizar. Es como si estuvieras intentando entender una canción nota por nota para ver cómo se compone.

Hablando de las aplicaciones del análisis de Fourier, eso sí que fue sorprendente para mí. Descubrir que se usa en tantas áreas, desde reducir vibraciones en edificios hasta procesar señales eléctricas, me hizo ver que aprender esto no es solo pasar una materia más de la carrera. Es algo que realmente tiene un impacto en el mundo real, en cosas que usamos y vemos todos los días.

Así que, aunque al principio pensaba que la serie de Fourier era solo otra cosa más que tenía que memorizar para el examen, ahora veo que es una herramienta súper útil. Me sorprendió aprender todo lo que se puede hacer con ella, y me hace querer saber más sobre cómo se aplican estas cosas en proyectos reales. Realmente, nunca pensé que las matemáticas que estamos aprendiendo pudieran tener tantas aplicaciones prácticas.Este proyecto nos llevó a trabajar con la función \(f(x)=x^4-5x^2-2x+1\), centrándonos en calcular y graficar sus coeficientes de Fourier en Excel. Desde la fase anterior, obtuvimos los coeficientes \(a_n\), \(b_n\), y \(a_0\) de forma algebraica, para comprender cómo se construye la serie de Fourier de esta función.

Luego, utilizamos Excel para representar gráficamente estos coeficientes. Esto nos permitió observar cómo cada término de la serie influye en la reconstrucción de la función original. Optamos por gráficos de puntos para mostrar los coeficientes \(a_n\) y \(b_n\). Esta etapa fue crucial porque nos ayudó a visualizar la relación entre los términos de la serie y la forma de la función.

La experiencia nos demostró la utilidad de Excel para analizar y comprender series matemáticas complejas. La representación gráfica de los coeficientes contribuyó significativamente a nuestra comprensión de la serie de Fourier y su aplicación en la representación matemática de funciones complejas.

Por lo que el proyecto representó una oportunidad para aplicar nuestros conocimientos matemáticos en un caso práctico, utilizando herramientas computacionales como Excel para simplificar y visualizar el trabajo.

El análisis y cálculo de la serie de Fourier para la función \(f(x)=x^4-5x-2x+1\) resultaron inicialmente intimidantes, pero al desglosar el proceso y entender cada paso, se evidenció la utilidad práctica de este método matemático en aplicaciones del mundo real. La paralelización del cálculo mediante bibliotecas específicas fue un giro determinante en nuestro proyecto, permitiéndonos manejar la computación de los coeficientes de Fourier de manera más eficiente y rápida.

El uso de la integración definida con el método del trapecio\cite*{khan-academy-regla-trapecio-no-date} para calcular cada término muy útil. Implementamos esta técnica en C, aprovechando los recursos del CPU usando técnicas de paralelización como la memoria compartida y semáforos para optimizar el tiempo de procesamiento. A través de la paralelización, pudimos realizar cálculos simultáneos, reduciendo significativamente la carga computacional que, de otro modo, habría sido monumental.

Posteriormente, la generación de archivos CSV contenía todos los datos numéricos necesarios para la representación gráfica. Aquí es donde intervinieron las bibliotecas de Python, matplotlib y pandas, que simplificaron la tarea de leer los datos y generar gráficos. Este proceso no solo facilitó la visualización de los resultados, sino que también permitió una interpretación más intuitiva de cómo cada coeficiente influencia la reconstrucción de la función original.

Después, en el caso del uso de hilos en vez de semáforos{[}25{]}, esta implementación ha resultado en una solución más sencilla y eficiente. Al dividir el cálculo de los coeficientes de Fourier entre múltiples hilos, el código logra una paralelización efectiva de la tarea, lo que conduce a una mejor utilización de los recursos del sistema y una ejecución más rápida. Además, al evitar la necesidad de gestionar la sincronización mediante semáforos u otras estructuras de control, se simplifica considerablemente la lógica del programa. En resumen, la estrategia de utilizar hilos ofrece una forma más clara y efectiva de abordar el problema, resultando en un código más legible, mantenible y eficiente.

Este proyecto no solo ha reforzado nuestros conocimientos y habilidades en matemáticas y programación, sino que también ha agudizado nuestra capacidad para aplicar estos conceptos en escenarios prácticos, preparándonos mejor para los desafíos del mundo real.

% Fase 5
MPI, al facilitar la paralelización de tareas, ha demostrado ser crucial en la computación de series de Fourier. Su capacidad para distribuir eficientemente la carga de trabajo entre múltiples procesadores acelera significativamente el cálculo de los coeficientes de Fourier, permitiendo manejar grandes volúmenes de datos con mayor rapidez y precisión.

Además, la visualización de las series de Fourier, apoyada en herramientas computacionales, es vital para observar cómo la suma de ondas simples se aproxima a la función original y cómo la cantidad de términos afecta la calidad de la representación. 

El uso de MPI no solo optimiza el tiempo de cálculo, sino que también permite explorar aplicaciones prácticas en ingeniería, como el procesamiento de señales y el análisis vibratorio, de manera más eficiente. Esta experiencia destaca la importancia de combinar el conocimiento teórico con herramientas avanzadas de computación para resolver problemas complejos en el mundo real.

En resumen, aunque el proceso manual refuerza la comprensión de los conceptos matemáticos, la implementación de MPI y otras herramientas computacionales es esencial para la eficiencia y precisión en el análisis de series de Fourier, ampliando las posibilidades de aplicación en diversas áreas de la ingeniería.

% Fase 6
A lo largo de esta práctica, la implementación y el análisis de las series de Fourier han sido una experiencia enriquecedora y transformadora. Las series de Fourier son una herramienta poderosa para analizar y representar funciones complejas mediante la descomposición en ondas sinusoidales. Aunque el cálculo manual puede resultar desafiante, ofrece una comprensión profunda de los fundamentos matemáticos que sustentan este método. Este enfoque meticuloso ha obligado a reflexionar cuidadosamente sobre cada paso del proceso, reforzando así mi comprensión de los conceptos matemáticos involucrados.

El uso de herramientas computacionales, como hojas de cálculo y programación, ha simplificado enormemente el proceso de cálculo y ha permitido trabajar con conjuntos de datos grandes. La implementación práctica de la serie de Fourier en C ha permitido experimentar con conceptos complejos de manera tangible, observando cómo se descompone una función en sus componentes básicos y luego se recompone para aproximar la función original. Además, al utilizar hilos y semáforos, se ha logrado distribuir la carga de trabajo y acelerar el cálculo de los coeficientes, demostrando la eficacia de estas herramientas en el análisis y la manipulación de funciones.

Comparar los cálculos manuales con los generados en C ha permitido apreciar las ventajas de la automatización y la precisión que ofrecen las herramientas computacionales. La visualización de las series de Fourier ha sido crucial para comprender su comportamiento y analizar su precisión, utilizando herramientas como Excel, Python y Google Sheets para graficar las series y observar cómo la suma de ondas simples se aproxima a la función original. La elección adecuada del rango para los ejes x e y ha sido esencial para visualizar correctamente la convergencia de la serie.

En la práctica, también se ha explorado el uso de MPI (Message Passing Interface) y CUDA (Compute Unified Device Architecture) para el cálculo de los coeficientes de las series de Fourier. MPI permite distribuir el cálculo en múltiples nodos, ideal para sistemas con varias máquinas trabajando en paralelo. Por otro lado, CUDA aprovecha la potencia de las GPU (Graphics Processing Units) para realizar cálculos de manera extremadamente rápida y eficiente. Utilizar CUDA ha demostrado una aceleración significativa en el cálculo de los coeficientes de Fourier, especialmente cuando se trata de grandes volúmenes de datos. Esta experiencia ha resaltado la capacidad de CUDA para manejar tareas computacionales intensivas con una velocidad impresionante, crucial en aplicaciones prácticas de ingeniería.

La implementación de la serie de Fourier utilizando CUDA ha sido particularmente reveladora. La capacidad de las GPU para manejar múltiples operaciones en paralelo ha permitido reducir drásticamente el tiempo de cálculo, esencial para trabajar con grandes conjuntos de datos de manera eficiente. La comparación entre MPI y CUDA ha destacado las fortalezas y debilidades de cada enfoque. Mientras que MPI es excelente para la distribución de tareas en un entorno de red, CUDA sobresale en la aceleración de cálculos en una única máquina con múltiples núcleos de procesamiento.

En resumen, la combinación de métodos manuales y computacionales para el análisis de series de Fourier no solo ha reforzado mi comprensión teórica, sino que también ha demostrado la importancia de las herramientas modernas en la resolución de problemas complejos. La experiencia práctica con MPI y CUDA ha ampliado mis habilidades en la programación paralela y distribuida, preparándome mejor para enfrentar desafíos en el campo de la ingeniería y la ciencia computacional. Esta práctica ha sido fundamental para mi desarrollo, subrayando la importancia de combinar conocimientos teóricos con habilidades prácticas en el uso de tecnologías en el ámbito del paralelismo.
\subsection{Juárez Tolamatl Oscar Uriel}

Es interesante indagar en la serie de Fourier, debido al gran impacto que ha tenido en estos tiempos modernos, aunque en un inicio no parece tan importante, cuando empiezas a adentrarte en su funcionamiento y sus aplicaciones las cosas cambian.

Una vez que aprendes sobre ello, te das cuenta que fue parte fundamental para el estudio de la computación moderna, en la medicina, la astronomía, la física en general, entre otras cosas, incluso en el estudio geológico.

Aunque en cierto momento puede parecer que la serie es un poco difícil de formular, lo cierto es que sin ella, los cálculos serían aún más complicadas, por lo que ésto viene a facilitar el estudio en muchos ámbitos, por ejemplo, el más común es el lograr almacenar el sonido dentro de una unidad de almacenamiento digital, tal fuera un disco duro, una memoria extraíble, entre todas.

Esto nos hace más fácil transferir información sonora y almacenarlo para después analizarlo de otra forma, extrayendo componentes que no sería fácil obtener sin ellas.

Es tan útil esta ecuación que puede ser utilizada para una ecuación cualquiera aunque no tenga absolutamente nada de información importante, como en el caso de la ecuación analizada en este documento, una ecuación obtenida de la nada, la cual se analiza dentro del tiempo y se traslada a la frecuencia.

Si es posible hacerlo con una ecuación nada importante, imaginar lo que se podría analizar dentro de una onda importante como las espaciales suena interesante, solo queda ver lo que es posible hacer y lo que no.

En la anterior fase del proyecto pudimos obtener la serie de Fourier de la ecuación correspondiente, pero solo fue de manera teórica; ahora, hemos podido observar el cómo es realmente esta serie después de ser evaluada desde los límites correspondientes.

La serie de fourier, como ya hemos investigado, es la discretización de una función continua para poder digitalizarla y manipularla con equipos digitales, con lo que la serie se vuelve un poco cuadrada, y cómo podemos comparar la serie continua original con la serie continua de fourier, es posible ver el cómo esta señal difiere un poco en el comportamiento de movimiento de la función, y como se mueve en el espacio dentro de los límites de \(-2 \pi\) hasta \(2 \pi\).

Aunque la manera de tabulación fue complicada debido a la cantidad de valores que se tomaron para n dentro de los límites, y después el sumar todos los valores para lograr tener una serie definida de Fourier, se pudo agilizar gracias a que se utilizó una herramienta externa como lo es Excel, el cual nos permite hacer cálculos complicados y tabulaciones si es que se ocupa de manera correcta, obteniendo una gran cantidad de valores que para una persona normal haciendo a mano le tomaría mucho más tiempo en realizar.

Finalmente, ver el cómo una función de transformación de una señal agiliza el proceso, junto con las herramientas que tenemos disponibles, nos hace pensar el cómo se realizaban estos cálculos antes de que existiera este método, la cantidad de tiempo que se invertía y todo el procedimiento tedioso por detrás, gracias a esta transformación que es la serie de Fourier es mucho más corto y fácil, con un resultado eficaz.

Esta fase 3 del proyecto fue interesante, pues en las anteriores partes hicimos primero el cálculo de la función de Fourier para nuestra ecuación correspondiente, y después graficamos la función original y la serie de Fourier correspondiente, con lo que pudimos comparar ambas graficaciones.

La serie de Fourier tiene diversas variables que se obtienen de acuerdo a los datos que poseemos, una de ellas es la variable "n" la cual representa el numero de veces que se dividirá el dominio de la ecuación para calcular los valores dentro de esos límites, en el caso de \(-2 \pi\) hasta \(2 \pi\) se divide entre 100, lo que crea que se deba calcular 100 veces el valor de las variables an y bn, lo cual es relativamente fácil pero tardado.

Gracias al computo paralelo, pudimos realizar en la totalidad los cálculos de cada una de las variables, lo cual nos ayuda a agilizar el proceso de obtener los valores de la serie buscados, para luego sumarlos y graficar cada uno de esos cien puntos.

Lo que un humano tardaría mucho tiempo y esfuerzo en calcular, lo hizo una maquina en poco tiempo, utilizando sus recursos disponibles pero obteniendo resultados favorables, todo gracias a la paralelización de procedimientos.

Con esto en mente, es posible traspasar tareas tardías a las máquinas para que las hagan en menor tiempo y tener resultados más rápido en menor tiempo, agilizando los procesos de cálculo dentro de un sistema, logrando un gran avance en los cálculos en escalas enormes.

Ahora se ha hecho con hilos, se nota la diferencia de ejecución, pues es rápido y se utilizan los recursos disponibles, es interesante ver la cantidad de maneras que se pueden realizar este tipo de cálculos.
% Fase 5
La resolución de este problema con MPI fue interesante, pues la ejecución fue diferente al asignar el número de subprocesos desde el inicio, lo que permite tener un control sobre las ejecuciones de los cálculos de la serie de Fourier, pues mil cálculos se hacen más rápido dividiéndolos entre seis en comparación al ejecutarse en uno solo, además, en comparación a los hilos, no es tan complejo el seccionado de subprocesos, pues en los hilos se deben de gestionar el número de hilos que se crean, mientras que en MPI es más fácil la gestión de subprocesos. 

Cada vez este proceso se va agilizando, con lo que la serie de Fourier se ha resuelto de una manera más sencilla, de modo que aprovechamos los recursos de una computadora para resolver estos procesos complejos al dividirlos en subproblemas más simples, con una ejecución más gestionable, de modo que se logre una buena gestión dentro de la resolución.

% Fase 6
El uso del entorno CUDA ha mejorado en gran parte el cálculo de esta función, pues con una CPU sólo se hacían 6 cálculos a la vez, pero con esta nueva herramienta se ha incrementado mucho la cantidad de núcleos de procesamiento, gracias a que una GPU posee mayor número de nucleos.

Paralelizar este proceso ha sido más fácil, pues al tener un mayor número de recursos disponibles, es posible agilizar el proceso de cálculo de los 1000 datos necesarios que necesitamos de la función, lo cual permite que sea más rapido en comparación a los anteriores métodos.

Comparandolo con el método anterior, el cual es MPI, la velocidad cambia drásticamente, pues el uso de la GPU y el uso de la CPU, al tener propósitos diferentes, cambian su arquitectura, lo cual hace que cada uno pueda usarse para paralelizar los procesos en casos diferentes.

La serie de fourier, aunque es extensa, no supone un problema al momento de calcular, por lo que una GPU, al poseer núcleos que no necesitan una cache propia, son perfectos para resolver este problema, lo que hace posible que nuestro cálculo sea efectuado.
\subsection{Jimeno Reyes Magaly Citlali }

Las series de Fourier son una herramienta poderosa para analizar y representar funciones. Aunque el cálculo manual puede ser desafiante, la experiencia ofrece una comprensión profunda de los fundamentos matemáticos. Las herramientas computacionales, como las hojas de cálculo y la programación, simplifican el proceso y permiten trabajar con conjuntos de datos grandes.

Las series de Fourier son una herramienta poderosa para analizar y representar funciones. Aunque el cálculo manual puede ser desafiante, la experiencia ofrece una comprensión profunda de los fundamentos matemáticos. Las herramientas computacionales, como las hojas de cálculo y la programación, simplifican el proceso y permiten trabajar con conjuntos de datos grandes, sin embargo, el proceso manual refuerza la comprensión de los conceptos matemáticos y la mecánica de las series de Fourier y permite desarrollar la capacidad de resolver problemas complejos de forma meticulosa y precisa.

Las series de Fourier tienen una amplia gama de aplicaciones en ingeniería, incluyendo:

Procesamiento de señales: Análisis y manipulación de señales de audio, video y otras.

Control de sistemas: Diseño de sistemas de control para robots, máquinas y otros dispositivos.

Análisis vibratorio: Estudio de las vibraciones en estructuras y máquinas.

La visualización de las series de Fourier es crucial para comprender su comportamiento y analizar su precisión. Permite observar cómo la suma de ondas simples se aproxima a la función original, y cómo la cantidad de términos afecta la calidad de la representación.

Algunas herramientas como Excel, Python y Google Sheets ofrecen herramientas básicas para graficar series de Fourier.

Es importante elegir un rango adecuado para el eje x y el eje y para visualizar la función original y la serie de Fourier con precisión, además mostrar diferentes gráficas con distintos números de términos para observar cómo la serie converge a la función original

Es importante destacar que la experiencia de escribir una serie de Fourier a mano, aunque desafiante, puede ser enriquecedora. Te obliga a pensar cuidadosamente en cada paso del proceso y te ayuda a desarrollar una comprensión profunda de las matemáticas involucradas. Sin embargo, para aplicaciones prácticas en el mundo real, las herramientas computacionales son invaluables. Te permiten trabajar con conjuntos de datos grandes de manera eficiente y precisa, lo que es esencial para resolver problemas complejos de ingeniería.

La implementación práctica de la serie de Fourier en C nos ha permitido experimentar con conceptos complejos de forma tangible. Hemos podido ver cómo la serie de Fourier puede descomponer una función en sus componentes básicos, las ondas sinusoidales, y luego recomponerla para aproximar la función original. El uso de hilos y semáforos ha permitido distribuir la carga de trabajo y acelerar el cálculo de los coeficientes de la serie de Fourier.

La visualización de las series de Fourier es crucial para comprender su comportamiento y analizar su precisión. Permite observar cómo la suma de ondas simples se aproxima a la función original, y cómo la cantidad de términos afecta la calidad de la representación. En nuestra implementación, hemos utilizado herramientas como: Python, C y Google Sheets para graficar las series de Fourier y observar su convergencia a la función original.

Nos ha permitido explorar las aplicaciones de las series de Fourier en ingeniería y comprender la importancia de la visualización en el análisis de funciones complejas. Aunque el proceso manual refuerza la comprensión de los conceptos matemáticos, las herramientas computacionales son esenciales para trabajar eficientemente con conjuntos de datos grandes y resolver problemas complejos en el mundo real.

Además en esta práctica podemos comparar los cálculos manuales con los generados en C. Por otro lado, al distribuir la carga de trabajo mediante hilos y semáforos, hemos logrado acelerar el cálculo de los coeficientes de la serie de Fourier, demostrando así la eficacia de las herramientas computacionales en el análisis y la manipulación de funciones.

% Fase 5

Las series de Fourier son una herramienta poderosa para analizar y representar funciones. Aunque el cálculo manual puede ser desafiante, la experiencia ofrece una comprensión profunda de los fundamentos matemáticos. Las herramientas computacionales, como las hojas de cálculo y la programación, simplifican el proceso y permiten trabajar con conjuntos de datos grandes.

Las series de Fourier son una herramienta poderosa para analizar y representar funciones. Aunque el cálculo manual puede ser desafiante, la experiencia ofrece una comprensión profunda de los fundamentos matemáticos. Las herramientas computacionales, como las hojas de cálculo y la programación, simplifican el proceso y permiten trabajar con conjuntos de datos grandes, sin embargo, el proceso manual refuerza la comprensión de los conceptos matemáticos y la mecánica de las series de Fourier y permite desarrollar la capacidad de resolver problemas complejos de forma meticulosa y precisa.

Las series de Fourier tienen una amplia gama de aplicaciones en ingeniería, incluyendo:

\begin{itemize}
	\item Procesamiento de señales: Análisis y manipulación de señales de audio, video y otras.
	\item Control de sistemas: Diseño de sistemas de control para robots, máquinas y otros dispositivos.
	\item Análisis vibratorio: Estudio de las vibraciones en estructuras y máquinas.
\end{itemize}

La visualización de las series de Fourier es crucial para comprender su comportamiento y analizar su precisión. Permite observar cómo la suma de ondas simples se aproxima a la función original, y cómo la cantidad de términos afecta la calidad de la representación.

Algunas herramientas como Excel, Python y Google Sheets ofrecen herramientas básicas para graficar series de Fourier. Es importante elegir un rango adecuado para el eje x y el eje y para visualizar la función original y la serie de Fourier con precisión, además de mostrar diferentes gráficas con distintos números de términos para observar cómo la serie converge a la función original.

Es importante destacar que la experiencia de escribir una serie de Fourier a mano, aunque desafiante, puede ser enriquecedora. Te obliga a pensar cuidadosamente en cada paso del proceso y te ayuda a desarrollar una comprensión profunda de las matemáticas involucradas. Sin embargo, para aplicaciones prácticas en el mundo real, las herramientas computacionales son invaluables. Te permiten trabajar con conjuntos de datos grandes de manera eficiente y precisa, lo que es esencial para resolver problemas complejos de ingeniería.

La implementación práctica de la serie de Fourier en C nos ha permitido experimentar con conceptos complejos de forma tangible. Hemos podido ver cómo la serie de Fourier puede descomponer una función en sus componentes básicos, las ondas sinusoidales, y luego recomponerla para aproximar la función original. El uso de hilos y semáforos ha permitido distribuir la carga de trabajo y acelerar el cálculo de los coeficientes de la serie de Fourier.

La visualización de las series de Fourier es crucial para comprender su comportamiento y analizar su precisión. Permite observar cómo la suma de ondas simples se aproxima a la función original, y cómo la cantidad de términos afecta la calidad de la representación. En nuestra implementación, hemos utilizado herramientas como: Python, C y Google Sheets para graficar las series de Fourier y observar su convergencia a la función original.

Nos ha permitido explorar las aplicaciones de las series de Fourier en ingeniería y comprender la importancia de la visualización en el análisis de funciones complejas. Aunque el proceso manual refuerza la comprensión de los conceptos matemáticos, las herramientas computacionales son esenciales para trabajar eficientemente con conjuntos de datos grandes y resolver problemas complejos en el mundo real.

Además, en esta práctica podemos comparar los cálculos manuales con los generados en C. Por otro lado, al distribuir la carga de trabajo mediante hilos y semáforos, hemos logrado acelerar el cálculo de los coeficientes de la serie de Fourier, demostrando así la eficacia de las herramientas computacionales en el análisis y la manipulación de funciones.

El desarrollo de código para calcular los valores de las tablas y la experiencia en la programación complementan este proceso, brindando una visión completa de cómo las series de Fourier pueden aplicarse en situaciones prácticas y cómo las herramientas computacionales pueden potenciar este análisis.

% Fase 6
A lo largo de las diferentes fases de esta práctica, la implementación y análisis de las series de Fourier ha sido una experiencia enriquecedora y transformadora. Las series de Fourier son una herramienta poderosa para analizar y representar funciones complejas a través de la descomposición en ondas sinusoidales. Aunque el cálculo manual puede ser desafiante, ofrece una comprensión profunda de los fundamentos matemáticos que sustentan este método. Este enfoque meticuloso me ha obligado a pensar cuidadosamente en cada paso del proceso, reforzando mi comprensión de los conceptos matemáticos involucrados.

El uso de herramientas computacionales, como hojas de cálculo y programación, ha simplificado enormemente el proceso de cálculo y ha permitido trabajar con conjuntos de datos grandes. En particular, la implementación práctica de la serie de Fourier en C nos ha permitido experimentar con conceptos complejos de manera tangible. Hemos podido ver cómo la serie de Fourier descompone una función en sus componentes básicos y luego la recompone para aproximar la función original. Además, al utilizar hilos y semáforos, hemos distribuido la carga de trabajo y acelerado el cálculo de los coeficientes, demostrando la eficacia de estas herramientas en el análisis y la manipulación de funciones.

Comparar los cálculos manuales con los generados en C nos ha permitido apreciar las ventajas de la automatización y la precisión que ofrecen las herramientas computacionales. Además, la visualización de las series de Fourier es crucial para comprender su comportamiento y analizar su precisión. Herramientas como Excel, Python y Google Sheets han sido fundamentales para graficar las series de Fourier y observar cómo la suma de ondas simples se aproxima a la función original. La elección adecuada del rango para los ejes x e y ha sido esencial para visualizar correctamente la convergencia de la serie.

En nuestra práctica, también hemos explorado el uso de MPI (Message Passing Interface) y CUDA (Compute Unified Device Architecture) para el cálculo de los coeficientes de las series de Fourier. MPI permite distribuir el cálculo en múltiples nodos, lo que es ideal para sistemas con múltiples máquinas trabajando en paralelo. Por otro lado, CUDA aprovecha la potencia de las GPU (Graphics Processing Units) para realizar cálculos de manera extremadamente rápida y eficiente. Al utilizar CUDA, hemos observado una aceleración significativa en el cálculo de los coeficientes de Fourier, especialmente cuando se trata de grandes volúmenes de datos. Esta experiencia ha demostrado la capacidad de CUDA para manejar tareas computacionales intensivas con una velocidad impresionante, lo que es crucial en aplicaciones prácticas de ingeniería.

La implementación de la serie de Fourier utilizando CUDA ha sido particularmente reveladora. La capacidad de las GPU para manejar múltiples operaciones en paralelo ha permitido reducir drásticamente el tiempo de cálculo, lo que ha sido esencial para trabajar con grandes conjuntos de datos de manera eficiente. La comparación entre MPI y CUDA ha resaltado las fortalezas y debilidades de cada enfoque. Mientras que MPI es excelente para la distribución de tareas en un entorno de red, CUDA sobresale en la aceleración de cálculos en una única máquina con múltiples núcleos de procesamiento.

En resumen, la combinación de métodos manuales y computacionales para el análisis de series de Fourier no solo ha reforzado mi comprensión teórica sino que también ha demostrado la importancia de las herramientas modernas en la resolución de problemas complejos. La experiencia práctica con MPI y CUDA ha ampliado mis habilidades en la programación paralela y distribuida, preparándome mejor para enfrentar desafíos en el campo de la ingeniería y la ciencia computacional. Esta práctica ha sido fundamental para mi desarrollo, subrayando la importancia de combinar conocimientos teóricos con habilidades prácticas en el uso de tecnologías en el ambito del paralelismo.

\subsection{Porra Zuñiga Braulio Gael}

Las series de Fourier son una herramienta poderosa para analizar y representar funciones. Aunque el cálculo manual puede ser desafiante, la experiencia ofrece una comprensión profunda de los fundamentos matemáticos. Las herramientas computacionales, como las hojas de cálculo y la programación, simplifican el proceso y permiten trabajar con conjuntos de datos grandes.

Las series de Fourier son una herramienta poderosa para analizar y representar funciones. Aunque el cálculo manual puede ser desafiante, la experiencia ofrece una comprensión profunda de los fundamentos matemáticos. Las herramientas computacionales, como las hojas de cálculo y la programación, simplifican el proceso y permiten trabajar con conjuntos de datos grandes, sin embargo, el proceso manual refuerza la comprensión de los conceptos matemáticos y la mecánica de las series de Fourier y permite desarrollar la capacidad de resolver problemas complejos de forma meticulosa y precisa.

La visualización de las series de Fourier es crucial para comprender su comportamiento y analizar su precisión. Permite observar cómo la suma de ondas simples se aproxima a la función original, y cómo la cantidad de términos afecta la calidad de la representación.

Algunas herramientas como Excel, Python y Google Sheets ofrecen herramientas básicas para graficar series de Fourier.

Es importante elegir un rango adecuado para el eje x y el eje y para visualizar la función original y la serie de Fourier con precisión, además mostrar diferentes gráficas con distintos números de términos para observar cómo la serie converge a la función original

Es importante destacar que la experiencia de escribir una serie de Fourier a mano, aunque desafiante, puede ser enriquecedora. Te obliga a pensar cuidadosamente en cada paso del proceso y te ayuda a desarrollar una comprensión profunda de las matemáticas involucradas. Sin embargo, para aplicaciones prácticas en el mundo real, las herramientas computacionales son invaluables. Te permiten trabajar con conjuntos de datos grandes de manera eficiente y precisa, lo que es esencial para resolver problemas complejos de ingeniería.

La implementación práctica de la serie de Fourier en C nos ha permitido experimentar con conceptos complejos de forma tangible. Hemos podido ver cómo la serie de Fourier puede descomponer una función en sus componentes básicos, las ondas sinusoidales, y luego recomponerla para aproximar la función original. El uso de hilos y semáforos ha permitido distribuir la carga de trabajo y acelerar el cálculo de los coeficientes de la serie de Fourier.

La visualización de las series de Fourier es crucial para comprender su comportamiento y analizar su precisión. Permite observar cómo la suma de ondas simples se aproxima a la función original, y cómo la cantidad de términos afecta la calidad de la representación. En nuestra implementación, hemos utilizado herramientas como: Python, C y Google Sheets para graficar las series de Fourier y observar su convergencia a la función original.

Nos ha permitido explorar las aplicaciones de las series de Fourier en ingeniería y comprender la importancia de la visualización en el análisis de funciones complejas. Aunque el proceso manual refuerza la comprensión de los conceptos matemáticos, las herramientas computacionales son esenciales para trabajar eficientemente con conjuntos de datos grandes y resolver problemas complejos en el mundo real.

Al comparar la evaluación de series de Fourier a mano con el uso de herramientas computacionales como hilos y procesos hijos, se destaca la diferencia en eficiencia y comprensión conceptual. Mientras que realizar cálculos a mano refuerza la comprensión de los fundamentos matemáticos y la mecánica subyacente de las series de Fourier, la programación con hilos y procesos hijos demuestra una eficacia superior en términos de velocidad y manejo de grandes conjuntos de datos. La aproximación a mano, aunque meticulosa y educativa, se ve superada por la automatización y la precisión que ofrecen los métodos computacionales. Sin embargo, este último requiere una comprensión previa de los conceptos matemáticos para una implementación correcta y eficiente, lo que nos lleva a concluir que la combinación de ambos enfoques podría proporcionar una base sólida para el aprendizaje académico y la aplicación práctica.

El uso de hojas de cálculo para la evaluación de la serie de Fourier ofrece una alternativa más accesible y menos técnica en comparación con la programación con hilos y procesos hijos. Aunque las hojas de cálculo pueden ser menos intimidantes y proporcionan una interfaz gráfica intuitiva para la visualización de resultados, suelen ser menos eficientes en el manejo de cálculos complejos y extensos. Por otro lado, los hilos y procesos hijos aprovechan la capacidad de procesamiento paralelo de las computadoras modernas, lo que resulta en un rendimiento significativamente mayor. Aunque la curva de aprendizaje para estas técnicas de programación es más pronunciada, el potencial de escalabilidad y la precisión que ofrecen son insuperables, especialmente cuando se trata de aplicaciones de ingeniería donde se manejan grandes volúmenes de datos y se requiere un alto grado de precisión. Al evaluar la efectividad y aplicabilidad de cada método, es crucial considerar el contexto educativo y las metas de aprendizaje. Trabajar a mano o con hojas de cálculo podría ser suficiente y más práctico. Sin embargo, para estudios avanzados o aplicaciones profesionales, especialmente en el campo de la ingeniería y la ciencia de datos, la programación con hilos y procesos hijos es claramente superior. No solo proporciona una profundidad de análisis y una eficiencia que los otros métodos no pueden igualar.

% Fase 5

OpenMPI es una biblioteca que permite la programación de aplicaciones paralelas en sistemas distribuidos. Implementar la resolución de la serie de Fourier con OpenMPI implicó dividir el trabajo entre varios nodos de computación. Cada nodo calculaba una parte de la serie, y luego los resultados se combinaban. Esta experiencia mostró cómo OpenMPI puede mejorar drásticamente el rendimiento en sistemas de computación de alto rendimiento (HPC), permitiendo resolver problemas complejos en tiempos significativamente reducidos.
La implementación del análisis de Fourier utilizando MPI y OpenMPI demostró ser una solución efectiva para manejar cálculos complejos y grandes volúmenes de datos. Desde las hojas de cálculo hasta la programación paralela avanzada, cada fase del proyecto destacó la importancia de las técnicas de programación paralela en la resolución eficiente de problemas científicos

% Fase 6

Las series de Fourier son una herramienta poderosa para analizar y representar funciones. Aunque el cálculo manual puede ser desafiante, la experiencia ofrece una comprensión profunda de los fundamentos matemáticos. Las herramientas computacionales, como las hojas de cálculo y la programación, simplifican el proceso y permiten trabajar con conjuntos de datos grandes, sin embargo, el proceso manual refuerza la comprensión de los conceptos matemáticos y la mecánica de las series de Fourier y permite desarrollar la capacidad de resolver problemas complejos de forma meticulosa y precisa. La visualización de las series de Fourier es crucial para comprender su comportamiento y analizar su precisión. Permite observar cómo la suma de ondas simples se aproxima a la función original, y cómo la cantidad de términos afecta la calidad de la representación. Algunas herramientas como Excel, Python y Google Sheets ofrecen herramientas básicas para graficar series de Fourier. Es importante elegir un rango adecuado para el eje x y el eje y para visualizar la función original y la serie de Fourier con precisión, además mostrar diferentes gráficas con distintos números de términos para observar cómo la serie converge a la función original.

Es importante destacar que la experiencia de escribir una serie de Fourier a mano, aunque desafiante, puede ser enriquecedora. Te obliga a pensar cuidadosamente en cada paso del proceso y te ayuda a desarrollar una comprensión profunda de las matemáticas involucradas. Sin embargo, para aplicaciones prácticas en el mundo real, las herramientas computacionales son invaluables. Te permiten trabajar con conjuntos de datos grandes de manera eficiente y precisa, lo que es esencial para resolver problemas complejos de ingeniería. La implementación práctica de la serie de Fourier en C nos ha permitido experimentar con conceptos complejos de forma tangible. Hemos podido ver cómo la serie de Fourier puede descomponer una función en sus componentes básicos, las ondas sinusoidales, y luego recomponerla para aproximar la función original. El uso de hilos y semáforos ha permitido distribuir la carga de trabajo y acelerar el cálculo de los coeficientes de la serie de Fourier. La visualización de las series de Fourier es crucial para comprender su comportamiento y analizar su precisión. Permite observar cómo la suma de ondas simples se aproxima a la función original, y cómo la cantidad de términos afecta la calidad de la representación. En nuestra implementación, hemos utilizado herramientas como: Python, C y Google Sheets para graficar las series de Fourier y observar su convergencia a la función original. Nos ha permitido explorar las aplicaciones de las series de Fourier en ingeniería y comprender la importancia de la visualización en el análisis de funciones complejas. Aunque el proceso manual refuerza la comprensión de los conceptos matemáticos, las herramientas computacionales son esenciales para trabajar eficientemente con conjuntos de datos grandes y resolver problemas complejos en el mundo real.

Al comparar la evaluación de series de Fourier a mano con el uso de herramientas computacionales como hilos y procesos hijos, se destaca la diferencia en eficiencia y comprensión conceptual. Mientras que realizar cálculos a mano refuerza la comprensión de los fundamentos matemáticos y la mecánica subyacente de las series de Fourier, la programación con hilos y procesos hijos demuestra una eficacia superior en términos de velocidad y manejo de grandes conjuntos de datos. La aproximación a mano, aunque meticulosa y educativa, se ve superada por la automatización y la precisión que ofrecen los métodos computacionales. Sin embargo, este último requiere una comprensión previa de los conceptos matemáticos para una implementación correcta y eficiente, lo que nos lleva a concluir que la combinación de ambos enfoques podría proporcionar una base sólida para el aprendizaje académico y la aplicación práctica. El uso de hojas de cálculo para la evaluación de la serie de Fourier ofrece una alternativa más accesible y menos técnica en comparación con la programación con hilos y procesos hijos. Aunque las hojas de cálculo pueden ser menos intimidantes y proporcionan una interfaz gráfica intuitiva para la visualización de resultados, suelen ser menos eficientes en el manejo de cálculos complejos y extensos. Por otro lado, los hilos y procesos hijos aprovechan la capacidad de procesamiento paralelo de las computadoras modernas, lo que resulta en un rendimiento significativamente mayor. Aunque la curva de aprendizaje para estas técnicas de programación es más pronunciada, el potencial de escalabilidad y la precisión que ofrecen son insuperables, especialmente cuando se trata de aplicaciones de ingeniería donde se manejan grandes volúmenes de datos y se requiere un alto grado de precisión. Al evaluar la efectividad y aplicabilidad de cada método, es crucial considerar el contexto educativo y las metas de aprendizaje. Trabajar a mano o con hojas de cálculo podría ser suficiente y más práctico. Sin embargo, para estudios avanzados o aplicaciones profesionales, especialmente en el campo de la ingeniería y la ciencia de datos, la programación con hilos y procesos hijos es claramente superior. No solo proporciona una profundidad de análisis y una eficiencia que los otros métodos no pueden igualar.

OpenMPI es una biblioteca que permite la programación de aplicaciones paralelas en sistemas distribuidos. Implementar la resolución de la serie de Fourier con OpenMPI implicó dividir el trabajo entre varios nodos de computación. Cada nodo calculaba una parte de la serie, y luego los resultados se combinaban. Esta experiencia mostró cómo OpenMPI puede mejorar drásticamente el rendimiento en sistemas de computación de alto rendimiento (HPC), permitiendo resolver problemas complejos en tiempos significativamente reducidos. La implementación del análisis de Fourier utilizando MPI y OpenMPI demostró ser una solución efectiva para manejar cálculos complejos y grandes volúmenes de datos. Desde las hojas de cálculo hasta la programación paralela avanzada, cada fase del proyecto destacó la importancia de las técnicas de programación paralela en la resolución eficiente de problemas científicos. La programación con Lenguaje C en la plataforma GNU-Linux de los cálculos de los términos de la serie de Fourier usando CUDA también ha mostrado ser una estrategia efectiva para aprovechar las capacidades de las GPU en el procesamiento paralelo masivo, permitiendo una aceleración significativa en la evaluación de la serie de Fourier. Esta implementación con CUDA no solo ha mejorado la eficiencia computacional, sino que también ha proporcionado una plataforma robusta para el análisis de funciones complejas en tiempo real.